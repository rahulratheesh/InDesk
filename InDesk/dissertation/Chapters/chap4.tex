% chap4.tex 

\chapter{InBlocks}\label{chap:inblocks}

\section{Google Blockly framework}
Blockly is a newer entrant into the world of Visual Programming Languages for the web as it was released in 2012. The core Graphical User interface of this environment is similar to Scratch, but also quite simplified and less cluttered. Users are presented with a Blocks pane in the left hand-side of the screen, where they can choose between different categories of block functions. The center of the screen, and the majority of the interface is devoted to the work area. This is basically a blank canvas where users can drag and drop blocks and connect them to form the structure of their code. In the lower right-hand corner of the screen is the trashbin, where users can drop chunks of blocks that they wish to delete. The upper side offers buttons to run the program.

Google Blockly is programmed entirely in Javascript. This makes integrating it within any webpage a very simple task, and also greatly simplifies any modifications made to the GUI. Furthermore, Blockly has integrated interpreters that translate the output of the block strucutures to Javascript, Python, or XML. This is a great advantage, since it allows for easy manipulation and adaptation of the user code to the robot. Finally, the default blocks available to the user can be easily customized using Block Factory. For this research, author has enhanced Google Blockly framework with a support for Android mobile devices.


